\documentclass[11pt, oneside]{article}   	% use "amsart" instead of "article" for AMSLaTeX format
\usepackage{geometry}                		% See geometry.pdf to learn the layout options. There are lots.
\geometry{letterpaper}                   		% ... or a4paper or a5paper or ... 
%\geometry{landscape}                		% Activate for for rotated page geometry
%\usepackage[parfill]{parskip}    		% Activate to begin paragraphs with an empty line rather than an indent
\usepackage{graphicx}				% Use pdf, png, jpg, or eps§ with pdflatex; use eps in DVI mode
								% TeX will automatically convert eps --> pdf in pdflatex		
\usepackage{amssymb}
\usepackage{natbib}
\usepackage{bm}

\title{Brief Article}
\author{The Author}
%\date{}							% Activate to display a given date or no date

\begin{document}
\maketitle
%\section{}
%\subsection{}


\section*{Introduction}\label{introduction}

My general research interests are in the field of spatial statistics
with natural science applications. Specifically, my thesis research
introduces local variable selection in spatially varying coefficient
regression models, and I am also working on the problem of spatial
confounding. In the sections below, I expand on these research problems,
my work on them, and my plans going forward.

\section*{Local variable selection}\label{local-variable-selection}

Whereas the coefficients in a typical spatial regression model are
constant over the model's spatial domain, the coefficients in a varying
coefficient regression (VCR) model are functions - here we'll assume
smooth functions - of the location \(s\) \citep{Hastie-Tibshirani-1993}.
The coefficient functions in a VCR model may be estimated by local
polynomial regression, a variant of kernel smoothing that uses Taylor's
expansion to represents the coefficient functions \(\bm{\beta}(s)\) as
polynomials in a neighborhood of \(s_0\). For instance, a local
polynomial model of order one estimates the value \(\bm{\beta}(s_0)\),
and the slope \(\bm{\beta}'(s_0)\) of the coefficient functions at
\(s_0\) \citep{Fan-Gijbels-1996,Fan-Zhang-1999}.

Prior research for VCR models has focused on global variable selection
for VCR models \citep{Wang-Li-Huang-2008,Wang-Xia-2009,Wei-Huang-Li-2011}. Global variable selection identifies the
covariates with nonzero coefficient functions and uses the same set of
covariates on the entire spatial domain. In contrast, my work shows how
to estimate which variables in a VCR model have nonzero coefficients at
any location in the model's domain. The method I developed is an
\(\mathcal{L}_1\) regularization procedure akin to the adaptive group
lasso \citep{Wang-Leng-2008}, so I call it local adaptive grouped
regularization (LAGR). Its ``oracle'' properties are the subject of a
manuscript that has been submitted to the Journal of the Royal
Statistical Society (Series B). In the manuscript and the accompanying
software package, the response of the regression model can follow any
exponential family distribution.

The estimation properties of the LAGR method are appealing, but due to
the \(\mathcal{L}_1\) regularization, the resulting estimator is a
nonlinear function of of the data and thus has complicated confidence
intervals \citep{Knight-Fu-2000}. A second manuscript, intended for
submission to the Journal of Agricultural, Biological, and Environmental
Statistics, demonstrates inference in the context of a VCR model
estimated by LAGR. Topics covered there are the degrees of freedom used
in estimation, estimating the AIC-optimal bandwidth and tuning
parameters, model averaging with the AIC, and using a parametric
bootstrap procedure to summarize quantities like confidence intervals
for local coefficients and the confidence that a local coefficient is
nonzero. Another manuscript, describing the R package {\tt lagr}, is in
preparation and intended for submission to the Journal of Statistical
Software.

There are several outstanding research problems in this area, such as
developing local variable selection in partially linear regression,
where some coefficients are constant and others vary. Another is
estimation and inference when the coefficient functions have different
degrees of smoothness, and when the coefficients are smoother in one
part of the domain than another.

\section*{Spatial confounding}\label{spatial-confounding}

A basic principle of spatial statistics is that nearby observations are
more more alike than distant ones. In regression models, this is true of
both the covariates and the response. In this setting, it can be unclear
whether the observed regression relationship is due to a genuine
relationship between the covariate and the response, or because they
both have spatial structure on the same observational units
\citep{Hodges-Reich-2010,Paciorek-2011}. This phenomenon is called
spatial confounding.

\cite{Paciorek-2011} focuses on the roles of left-out confounding variables
and the spatial scale of variation in geostatistical models, concluding
that estimators can be biased by confounders that vary on a smaller
spatial scale than the observed covariates. On the other hand,
\cite{Hodges-Reich-2010} argue that confounding can result from the
arrangement of observational units in an intrinsic conditionally
autoregressive (ICAR) model \citep{Besag-York-Mollie-1991}. They show that
effects reported as significant in the scientific literature may
disappear completely when the response is projected orthogonal to the
ICAR adjacency matrix.

Disagreement in the literature and the relevance to fields employing
spatial statistical methods suggest that there is productive research to
be done here. For instance, I am currently studying methods to decompose
variation in the response of an ICAR model into unconfounded, residual,
and potentially confounded components. Another track I am exploring is
how to discern confounding by aggregating or subdividing the data to
vary the spatial scale of variation.

\section*{References}
\bibliographystyle{plain}
\bibliography{../references}

\end{document}  